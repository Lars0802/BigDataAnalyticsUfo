\title{Bericht der Gruppe \enquote{Ufo}}
\subtitle{Eine Analyse der Abhängigkeit der Wetters auf vermeintliche Ufo-Sichtungen.}         % Optional.

\keywords{Übung \enquote{Big Data Analytics}, Sommersemester 2021, Ufo}

\author{Lars Thomsen, Uzeyir Mammadov}
\affiliation{%
    \institution{%
        Martin-Luther-Universität Halle-Wittenberg
    }
}
\email{[lars.thomsen][uzeyir.mammadov]@student.uni-halle.de}

\begin{abstract}
    Mit den immer besser und intelligenter werdenden Technologien lassen sich heutzutage viel mehr astronomische Forschungen durchführen als früher. Damit rückt auch die Antwort auf die Frage, ob \enquote{wir} im Universum alleine sind, in greifbare Nähe. Der erste Gedanke fällt dabei auf die in der Vergangenheit und in der Gegenwart gesichteten vermeintlich fremden Flugkörper. In unserem Projekt haben wir uns mit der Fragestellung beschäftigt, ob das Wetter einen Einfluss auf eine Ufo-Sichtung hat. Dabei wurden historische Ufo-Sichtungen mit den dazu passenden Wetterdaten von der Stelle der Ufo-Sichtung verglichen und analysiert. Als Ergebnis konnte eine leichte Abhängigkeit zwischen dem an dem Zeitpunkt herrschenden Wetter und der Anzahl der Ufo-Sichtungen festegestellt werden. An Tagen mit viel Sonne und gutem Wetter wurden überdurchschnittlich viele Ufos gesichtet.

\end{abstract}

\maketitle
