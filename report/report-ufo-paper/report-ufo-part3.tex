\section{Die ersten Versuche} \label{versuch1}  % Change title accordingly

Die erste Idee war es, die Wetterdaten für unsere Forschungsfrage von den \enquote{National Centers for Enviromental Information} (NCEI) zu beziehen. Diese Datensätze beinhalten für unsere Analyse zwei wesentliche Attribute: \code{HourlySkyCondition} und \code{HourlyVisibility}. Diese beiden Werte beschreiben zum Ersten den klassifizierten Zustand des Himmels (z.B. Regenwolken, Schleierwolken, Nebel etc.) und zum Zweiten die prozentuale Bedecktheit des Himmels. Die Liste aller verfügbaren Stationen sind ebenfalls beim NCEI verfügbar(Link Stationen).

Um die Daten einer Ufo-Sichtung mit den dazugehörigen Wetterdaten zu verknüpfen, wird das jeweilige eindeutige \code{city, state}-Tupel der am nahestehendsten Wetterstation zugeordnet. Um die Ergebnisse nicht zu sehr zu verfälschen, wurde das zulässige Einzugsgebiet einer Wetterstation auf 20 Kilometer, mit der Station als Zentrum, begrenzt.

Diese Idee konnte leider nicht umgesetzt werden. Es standen zwar zwei Versionen der API vom NCEI zur verfügung, allerdings greifen diese auf unterschiedliche Stationen zurück. Dabei waren zwei Probleme die Hauptursache für das Aufgeben dieser ersten Idee: Zum Ersten waren die internen Bezeichnungen der Stationen bei Version~1 der API verschieden zu denen der uns zur verfügung stehenden Liste. Eine einheitliche Liste aller Stationen, welche durch Version~1 der API bedient werden war nicht verfügbar. Zum Zweiten wurde von Version~2 der API eine verschiedene zur bereits verarbeiteten Liste an Stationen verwendet, welche nur unter Umständen einsehbar war. Somit war es nicht möglich mit den verwendeten Datensätzen die gestellte Forschungsfrage zu bearbeiten.