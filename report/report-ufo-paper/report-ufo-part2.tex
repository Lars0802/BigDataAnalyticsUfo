\section{Daten} \label{data}

Die verwendeten Datensätze stammen aus einer Datenbank vom \enquote{National UFO Reporting Center} (NUFORC). In diese Datenbank kann jede Person ihre vermeintliche Ufo-Sichtung -- entweder online über ein Formular oder per Telefon -- eintragen. Des Weiteren senden gewisse Messstationen (z.B. MOAfyxcc) auffällige Messdaten automatisch an die Datenbank. Um offensichtlichen Fakes entgegenzuwirken werden die eingereichten Daten vor der monatlichen Veröffentlichung von den Betreibern des Portals überprüft.

Jeder Eintrag des Datensatzes besteht aus dem Datum und der Uhrzeit der Sichtung, der Stadt sowie dem Kürzel des dazugehörigen Bundesstaates (ausschließlich für die USA), einer klassifizierten Beschreibung der Form des Ufos und gegebenenfalls einer kurzes Zusammenfassung und Beschreibung aus der Sicht des Einsenders.

Für diese Daten haben wir eine einfache Datenbank erstellt, um auch offline auf diese zugreifen zu können. Um die Daten einfacher verarbeiten zu können, bereiten wir diese während der Abfrage von unserer Datenbank auf und treffen eine Vorauswahl an brauchbaren Daten. Die Kriterien für die Vorauswahl wurden durch stichprobenartige Abfragen festgelegt. Dem entsprechend werden Einträge von vorhin beschriebenen externen Messstationen, Einträge mit fehlenden Inhalten oder \enquote{?} als Inhalt nicht betrachtet. Der Datentyp aller Attribute ist der Einfachheit halber nur string.Zu einem späteren Zeitpunkt werden Datum und Uhrzeit in ein datetime-Format überführt. Mit dem Wissen aus den stichprobenartigen Abfragen legen wir und auf die zwei gängigsten Formate fest -- 'm/d/y H:M' und 'm/d/y'. Andere in dem Datensatz vorkommenden Formate oder von den Einsendern eigenständige Angaben beachten wir nicht. Unsere aufbereiteten Daten werden letztendlich durch das Tripel \code{datetime}, \code{city} und \code{state} beschrieben.