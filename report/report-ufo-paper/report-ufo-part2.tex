\section{Daten} \label{data}

Die verwendeten Datensätze stammen aus einer Datenbank vom \enquote{National UFO Reporting Center} (NUFORC). In diese Datenbank kann jede Person ihre vermeintliche Ufo-Sichtung -- entweder online über ein Formular oder per Telefon -- eintragen. Des Weiteren senden gewisse Messstationen, zum Beispiel MADAR Nodes, auffällige Messdaten automatisch an die Datenbank\cite{madar:2020}. Da die zur verfügung stehenden Daten für das Projekt keinen Mehrwert baten, wurden diese nicht beachtet. Um offensichtlichen Fakes entgegenzuwirken werden die eingereichten Daten vor der monatlichen Veröffentlichung von den Betreibern des Portals überprüft.

Jeder Eintrag des Datensatzes besteht aus dem Datum und der Uhrzeit der Sichtung, der Stadt sowie dem Kürzel des dazugehörigen Bundesstaates (ausschließlich für die USA), einer klassifizierten Beschreibung der Form des Ufos und gegebenenfalls einer kurzen Zusammenfassung und Beschreibung aus der Sicht des Einsenders.

Für diese Daten haben wir eine einfache Datenbank erstellt, um auch offline auf diese zugreifen zu können. Um die Daten einfacher verarbeiten zu können, bereiten wir diese während der Abfrage von unserer Datenbank auf und treffen eine Vorauswahl an brauchbaren Daten. Die Kriterien für die Vorauswahl wurden durch stichprobenartige Abfragen festgelegt. Dem entsprechend werden Einträge von vorhin beschriebenen externen Messstationen, Einträge mit fehlenden Inhalten oder \enquote{?} als Inhalt nicht betrachtet. Der Datentyp aller Attribute ist der Einfachheit halber nur string. Zu einem späteren Zeitpunkt werden Datum und Uhrzeit in ein datetime-Format überführt. Mit dem Wissen aus den stichprobenartigen Abfragen legen wir uns auf die zwei gängigsten Formate fest -- 'm/d/y H:M' und 'm/d/y'. Andere in dem Datensatz vorkommenden Formate oder von den Einsendern eigenständige Angaben beachten wir nicht. Die aufbereiteten Daten werden letztendlich durch das Tripel \code{datetime}, \code{city} und \code{state} beschrieben.

Als Quelle für die Wetterdaten haben wir uns letztendlich für \enquote{Meteostat} entschieden\cite{meteostat:2021}. Auf diese Daten wird über die dazugehörige Python-Library zugegriffen. Essenziell für unser Projekt sind die Attribute \code{tsun} (Anzahl der Sonnenminuten) und \code{coco} (Klassifizierter Zustand des Himmels). Die Daten sind für unsere Vorhaben bereits von ausreichender Qualität, sodass in diesem Fall keine weitere Aufbereitung und Säuberung der Daten nötig ist.

\begin{table}[t]
    \caption{Kennzahlen des Datensatzes.}
    \label{tab:data}
    \centering
    \small
    \begin{tabular}{l r}
        \toprule
        Einträge & Anzahl\\
        \midrule
        Gesamt & 96~924\\
        Davon einzigartige Orte & 25~234\\
        \midrule
        Orte mit Wetterdaten, gesamt & 2~020\\
        Davon mit Sonnenminuten pro Stunde & 26\\
        Davon mit Sonnenminuten pro Tag & 116\\
        Davon mit condition codes & 1~886\\
        \bottomrule
    \end{tabular}
\end{table}