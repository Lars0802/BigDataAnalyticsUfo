\section{Einleitung} \label{einleitung}

Auf den ersten Blick klingen vermeintliche Ufo-Sichtungen immer nach an den Haaren herbeigezogenen Erfindungen und Lügengeschichten. Doch wie viel Wahrheit steckt in diesen Sichtungen, und lassen sich gegebenenfalls Muster in diesen erkennen?

% Welchen Einfluss nimmt das zum Zeitpunkt der Sichtung herrschende Wetter auf die Sichtung von vermeintlichen Ufos?
Im Zuge unseres Projektes haben wir uns folgende Forschungsfrage gestellt: Existieren Abhängigkeiten zwischen vermeintlichen Ufo-Sichtungen, und welchen Einfluss nimmt das zum Zeitpunkt der Sichtung herrschende Wetter?

Als Grundlage dafür dienen uns ein Datensatz des \enquote{National UFO Reporting Center} (NUFORC)\cite{nuforc:2021} sowie Wetterdaten für Vergleiche und Analysen von Meteostat\cite{meteostat:2021}.

Der Bericht gliedert sich wie folgt: Kapitel \ref{data} beschreibt die verwendeten Daten und Datensätze. In Kapitel \ref{versuch1} werden die ersten Versuche kurz erklärt, bevor in Kapitel \ref{forschungsfrage} die Forschungsfrage und ihre Bearbeitung vorgestellt wird. Kapitel \ref{evaluation} diskutiert die Ergebnisse unserer Frage. Im Schlusskapitel (\ref{fazit}) geben wir unser Fazit und runden den Bericht ab.

\begin{figure}[t]
    \centering
    \includegraphics[width=\columnwidth]{nuforc_logo}
    \caption{NUFORC.}
    \label{fig:nuforc}
\end{figure}